%-----------------------------------------------------------------------
% Beginning of chap1.tex
%-----------------------------------------------------------------------
%
%  AMS-LaTeX sample file for a chapter of a monograph, to be used with
%  an AMS monograph document class.  This is a data file input by
%  chapter.tex.
%
%  Use this file as a model for a chapter; DO NOT START BY removing its
%  contents and filling in your own text.
% 
%%%%%%%%%%%%%%%%%%%%%%%%%%%%%%%%%%%%%%%%%%%%%%%%%%%%%%%%%%%%%%%%%%%%%%%%

\section*{Questions and Answers}

\noindent {\em Q}: Which lectures are relevant for the test?\\

\noindent {\em A}: Weeks 1-17. \\

\noindent {\em Q}: What do we have to know?\\

\noindent {\em A}: The following {\em will not} be asked, though the statements should be known:
\begin{itemize}
\item Details of step-length selection for gradient descent.
\item Proofs in Section 4.2 on the convergence of gradient descent. What needs to be remembered from this section is that gradient descent has {\em linear} convergence (implied by Theorem 4.4).
\item Proof of Theorem 5.7 (quadratic convergence of Newton's method), though the statement should be known.
\item Proof of Lemma 6.8.
\item Theorem 12.1 (proof of convergence for interior point methods)
\item Proof of Theorem 15.1
\item Derivation of KKT conditions
\item Proof of Theorem 17.2
\item Goemans-Williamson and \textsc{MaxCut} (Lecture 18)
\end{itemize}
As a rule of thumb, it is beneficial for the understanding to know the proof ideas, though precise reconstruction of proofs will not be asked.\\


\noindent {\em Q}: Which problems are relevant?\\

\noindent {\em A}: Problem that use MATLAB or require cutting a shape out of paper will not be relevant to the test.


%-----------------------------------------------------------------------
% End of chap1.tex
%-----------------------------------------------------------------------
