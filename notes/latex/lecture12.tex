%-----------------------------------------------------------------------
% Beginning of chap2.tex
%-----------------------------------------------------------------------
%
%  AMS-LaTeX sample file for a chapter of a monograph, to be used with
%  an AMS monograph document class.  This is a data file input by
%  chapter.tex.
%
%  Use this file as a model for a chapter; DO NOT START BY removing its
%  contents and filling in your own text.
% 
%%%%%%%%%%%%%%%%%%%%%%%%%%%%%%%%%%%%%%%%%%%%%%%%%%%%%%%%%%%%%%%%%%%%%%%%


\chapter*{Lecture 12}
\addcontentsline{toc}{chapter}{Lecture 12}
\setcounter{chapter}{12}
\setcounter{section}{0}
\setcounter{equation}{0}
\setcounter{theorem}{0}
%\numberwithin{section}{chapter}
\numberwithin{equation}{chapter}
\numberwithin{theorem}{chapter}

%\epigraph{}{--- \textup{}}

Recall the following primal dual standard forms of linear programming,
\begin{equation}\label{eq:standard}\tag{P}
 \minimize \ip{\vct{c}}{\vct{x}} \quad \subjto \mtx{A}\vct{x}=\vct{b}, \ \vct{x}\geq \zerovct.
\end{equation}
for a matrix $\mtx{A}\in \R^{m\times n}, \vct{b}\in \R^m$ and $\vct{c}\in \R^n$, 
and
\begin{equation}\label{eq:dual}\tag{D}
\maximize \ip{\vct{b}}{\vct{y}} \quad \subjto \mtx{A}^{\trans}\vct{y}+\vct{s}=\vct{c}, \ \vct{s}\geq \zerovct.
\end{equation}
For what follows, we assume $m\leq n$, as otherwise~\eqref{eq:dual} is unbounded and~\eqref{eq:standard} empty.

Based on~\eqref{eq:standard} and~\eqref{eq:dual}, we get the optimality conditions (see Lecture 11)
\begin{align}\label{eq:opt2}\tag{O}
 \begin{split}
  \mtx{A}^{\trans}\vct{y}+\vct{s}-\vct{c} &= \zerovct\\
  \mtx{A}\vct{x}-\vct{b} & = \zerovct\\
  \mtx{X}\mtx{S}\vct{e} &= \zerovct\\
  \vct{x}&\geq \zerovct\\
  \vct{s}&\geq \zerovct,
 \end{split}
\end{align}
where $\vct{X}$ is the diagonal matrix with the $x_i$ in the diagonal, $\mtx{S}$ the diagonal matrix with the $s_i$ on the diagonal, and $\vct{e}$ the vector with all ones ($\mtx{X}\mtx{S}\vct{e}$ is just a compact way of writing the vector with entries $x_is_i$).
If we define the function $\R^{2n+m}\to \R^{2n+m}$
\begin{equation}\label{eq:F}
 F(\vct{x},\vct{y},\vct{s}) = \begin{pmatrix}
                               \mtx{A}^{\trans}\vct{y}+\vct{s}-\vct{c}\\
                               \mtx{A}\vct{x}-\vct{b}\\
                               \mtx{X}\mtx{S}\vct{e} 
                              \end{pmatrix}
\end{equation}
then what we are looking for is a {\em root} $(\vct{x}^*,\vct{y}^*,\vct{s}^*)$ of this function, i.e., a point with $F(\vct{x}^*,\vct{y}^*,\vct{s}^*)=\zerovct$, with the additional property that $\vct{x}^*$ and $\vct{s}^*$ be non-negative. For later reference, we record the form of the {\em Jacobian} matrix of $F$,

\begin{equation*}
 \mtx{J}F(\vct{x},\vct{y},\vct{s}) = \begin{pmatrix}
\zerovct & \mtx{A}^{\trans} & \mtx{I} \\
  \mtx{A} & \zerovct & \zerovct \\
  \mtx{S} & \zerovct & \mtx{X}
 \end{pmatrix}.
\end{equation*}

\section{Towards an efficient interior-point method}
The challenge is to find an iterative method of solving this rootfinding problem while preserving the non-negativity constraints. We discuss three approaches, in increasing order of sophistication. The third of these approaches forms the basis of {\em primal dual interior point methods}. 

\subsection{A first attempt}
A first attempt at solving the problem $F(\vct{x},\vct{y},\vct{s})=\zerovct$, with $\vct{x}\geq \zerovct, \vct{s}\geq \zerovct$, is to apply Newton's method with really small steps.
For that, at each step we solve
\begin{equation*}
\begin{pmatrix}
  \zerovct & \mtx{A}^{\trans} & \mtx{I} \\
  \mtx{A} & \zerovct & \zerovct \\
  \mtx{S}^{(k)} & \zerovct & \mtx{X}^{(k)}
 \end{pmatrix}
\begin{pmatrix} \Delta\vct{x}\\ \Delta \vct{y}\\ \Delta\vct{s} \end{pmatrix} = \begin{pmatrix} \vct{c}-\vct{s}^{(k)}-\mtx{A}^{\trans}\vct{y}^{(k)}\\ \vct{b}-\mtx{A}\vct{x}^{(k)}\\ -\vct{X}^{(k)}\mtx{S}^{(k)}\vct{e}\end{pmatrix}.
\end{equation*}
and then compute
\begin{equation*}
\begin{pmatrix} \vct{x}^{k+1}\\ \vct{y}^{k+1}\\ \vct{s}^{k+1}\end{pmatrix}= \begin{pmatrix} \vct{x}^{k}\\ \vct{y}^{k}\\ \vct{s}^{k}\end{pmatrix} +\alpha_k \begin{pmatrix} \Delta\vct{x}\\ \Delta \vct{y}\\ \Delta\vct{s} \end{pmatrix},
 \end{equation*}
 choosing the step length $\alpha_k$ so that the non-negativity of $\vct{x}$ and $\vct{s}$ remains. Note that above we have a + sign in front of alpha. That is because in the system of equations above, we took the negative $-F(\vct{x}_k,\vct{y}_k,\vct{s}_k)$. 

 \begin{example}
  Consider the linear programming problem
  \begin{align*}
   \minimize & x_1+2x_2-2x_3\\
   & x_1-2x_3\\
   & x_2-x_3=-1\\
   & x_1\geq 0, \ x_2\geq 0, \ x_3\geq 0.
  \end{align*}
The data for this problem is given by
\begin{equation*}
 \mtx{A} = \begin{pmatrix} 1 & 0 & -2\\ 0 & 1 & -1\end{pmatrix}, \ 
 \vct{b} = \begin{pmatrix} 1 \\ -1 \end{pmatrix}, \ 
 \vct{c} = \begin{pmatrix} 1 \\ 2 \\ -2 \end{pmatrix}.
\end{equation*}
Based on this data, the function $F$ given by
\begin{equation*}
 F(\vct{x},\vct{y},\vct{s}) = \begin{pmatrix}
                               y_1+s_1-1\\
                               y_2+s_2-2\\
                               -2y_1-y_2+s_3+2\\
                               x_1-2x_3-1\\
                               x_2-x_3+1\\
                               x_1s_1\\
                               x_2s_2\\
                               x_3s_3
                              \end{pmatrix}
\end{equation*}
and the Jacobian matrix
\begin{equation*}
 \mtx{J}F(\vct{x},\vct{y},\vct{s}) = \begin{pmatrix}
                                0 & 0 & 0 & 1 & 0 & 1 & 0 & 0\\
                                0 & 0 & 0 & 0 & 1 & 0 & 1 & 0\\
                                0 & 0 & 0 & -2 & -1 & 0 & 0 & 1\\
                                1 & 0 & -2 & 0 & 0 & 0 & 0 & 0 \\
                                0 & 1 & -1 & 0 & 0 & 0 & 0 & 0 \\
                                s_1 & 0 & 0 & 0 & 0 & x_1 & 0 & 0 \\
                                0 & s_2 & 0 & 0 & 0 & 0 & x_2 & 0\\
                                0 & 0 & s_3 & 0 & 0 & 0 & 0 & x_3
                               \end{pmatrix}.
\end{equation*}
With this data at hand, we can easily use a Python program to solve the problem for us, making sure that the steplength is small enough. Starting with $\vct{x}^{(0)}=\vct{s}^{(0)}=(1,1,1)^\trans$, we get the sequence of step lengths
\begin{equation*}
 \alpha_0 = 0.5455, \ \alpha_1=0.5455, \ \alpha_2 = 1,
\end{equation*}
with the corresponding sequence of $\vct{x}$ vectors
\begin{equation*}
 \vct{x}^{(1)} = \begin{pmatrix}
                  1.1818\\ 0\\ 0.5455
                 \end{pmatrix}, \ 
 \vct{x}^{(2)} = \begin{pmatrix}
                  2.1736\\ 0 \\ 0.7934
                 \end{pmatrix}, \
 \vct{x}^{(3)} = \begin{pmatrix}
                  3 \\ 0 \\ 1
                 \end{pmatrix}.
\end{equation*}
One verifies that $\vct{x}^{(3)}$ is in fact a vertext of the feasible set, and $\ip{\vct{x}^{(3)}}{\vct{c}}=1$ gives the optimal value. 
 \end{example}

\subsection{A second attempt}
In general, the method of choosing small step lengths can be slow. A variation would be to solve for
\begin{equation}\label{eq:ap2}
 F(\vct{x},\vct{y},\vct{s}) = \begin{pmatrix}
                               \vct{0}\\ \vct{0} \\ \tau \vct{e}
                              \end{pmatrix},
\end{equation}
for a parameter $\tau>0$. Given some fixed initial values $\vct{x}\geq 0$ and $\vct{s}\geq 0$, compute the {\em duality measure}
\begin{equation*}
 \mu = \frac{1}{n} \sum_{i=1}^n x_is_i
\end{equation*}
as the average of the products and using the {\em centering parameter} $\sigma\in (0,1)$, set $\tau = \sigma \mu$. When aiming to solve~\eqref{eq:ap2} using Newton's method, we are aiming towards a solution $(\vct{x}^*,\vct{y}^*,\vct{s}^*)$ where instead of 
asking that $x_i^*s_i^*=0$, we want that $x_i^*s_i^*=\sigma \mu$, a value that is strictly positive, but smaller than average of the initial value.

In an additional twist to this approach, after starting with an initial guess $(\vct{x},\vct{y},\vct{s})$, we perform only {\em one} Newton step, and then update the duality measure $\mu$ with the new values of $\vct{x}$ and $\vct{s}$. This way we arrive at the following algorithm: 
\begin{itemize}
 \item Start with $(\vct{x}^{(0)},\vct{y}^{(0)},\vct{s}^{(0)})$;
 \item For each $k\geq 0$, compute 
 \begin{equation*}
  \mu^{(k)} = \frac{1}{n} \sum_{i=1}^n x_is_i
 \end{equation*}
and $\sigma_k$. Solve
\begin{equation*}
 \begin{pmatrix}
  \zerovct & \mtx{A}^{\trans} & \mtx{I} \\
  \mtx{A} & \zerovct & \zerovct \\
  \mtx{S}^{(k)} & \zerovct & \mtx{X}^{(k)}
 \end{pmatrix}
\begin{pmatrix} \Delta\vct{x}\\ \Delta \vct{y}\\ \Delta\vct{s} \end{pmatrix} = \begin{pmatrix} \vct{c}-\vct{s}^{(k)}-\mtx{A}^{\trans}\vct{y}^{(k)}\\ \vct{b}-\mtx{A}\vct{x}^{(k)}\\ -\vct{X}^{(k)}\mtx{S}^{(k)}\vct{e}+\sigma \mu^{(k)}\vct{e}\end{pmatrix}
\end{equation*}
and compute
\begin{equation*}
\begin{pmatrix} \vct{x}^{k+1}\\ \vct{y}^{k+1}\\ \vct{s}^{k+1}\end{pmatrix}= \begin{pmatrix} \vct{x}^{k}\\ \vct{y}^{k}\\ \vct{s}^{k}\end{pmatrix} +\alpha_k \begin{pmatrix} \Delta\vct{x}\\ \Delta \vct{y}\\ \Delta\vct{s} \end{pmatrix},
 \end{equation*}
for a small enough $\alpha_k>0$ to ensure non-negativity.
\end{itemize}

\subsection{A third attempt}
For the purpose of analysis, it is convenient to let an interior point method operate on vectors that satisfy the first two equalities in~\eqref{eq:opt2} exactly. Define the feasible and strictly feasible sets as
\begin{align*}
 \mathcal{F} &= \{(\vct{x},\vct{y},\vct{s}) \mid \mtx{A}^{\trans}\vct{y}+\vct{s}=\vct{c}, \ \mtx{A}\vct{x}=\vct{b}, \ \vct{x}\geq \zerovct, \ \vct{s}\geq \zerovct\}\\
 \mathcal{F}^{\circ} &= \{(\vct{x},\vct{y},\vct{s}) \mid \mtx{A}^{\trans}\vct{y}+\vct{s}=\vct{c}, \ \mtx{A}\vct{x}=\vct{b}, \ \vct{x}> \zerovct, \ \vct{s}> \zerovct\}\\
\end{align*}
Restricting to points in $\mathcal{F}^{\circ}$, the computation of the Newton update in the second approach would change to
\begin{equation*}
 \begin{pmatrix}
  \zerovct & \mtx{A}^{\trans} & \mtx{I} \\
  \mtx{A} & \zerovct & \zerovct \\
  \mtx{S}^{(k)} & \zerovct & \mtx{X}^{(k)}
 \end{pmatrix}
\begin{pmatrix} \Delta\vct{x}\\ \Delta \vct{y}\\ \Delta\vct{s} \end{pmatrix} = \begin{pmatrix} \zerovct \\ \zerovct \\ -\vct{X}^{(k)}\mtx{S}^{(k)}\vct{e}+\sigma \mu^{(k)}\vct{e}\end{pmatrix}
\end{equation*}

In each iteration, a Newton step is taken in the direction of the {\em central path}. This is a curve in $\mathcal{F}^{\circ}$ defined as the set of solutions of
\begin{align}\label{eq:cpopt}
 \begin{split}
  \mtx{A}^{\trans}\vct{y}+\vct{s}-\vct{c} &= \zerovct\\
  \mtx{A}\vct{x}-\vct{b} & = \zerovct\\
  \mtx{X}\mtx{S}\vct{e} &= \tau \vct{e}\\
  \vct{x}&> \zerovct\\
  \vct{s}&> \zerovct,
 \end{split}
\end{align}
where $\tau>0$. As $\tau\to 0$, any solution of this system will converge to an optimal primal-dual vector $(\vct{x},\vct{y},\vct{s})$ for the original linear programming problem. As we will see, practical primal-dual interior point methods will try to ensure that we always move within a neighbourhood of the central path. While this third approach lends itself well to analysis, one problem is that a starting point $(\vct{x}^{(0)},\vct{y}^{(0)},\vct{s}^{(0)})\in \mathcal{F}^{\circ}$ may be hard to find.



% %-----------------------------------------------------------------------
% % End of chap1.tex
% %-----------------------------------------------------------------------
