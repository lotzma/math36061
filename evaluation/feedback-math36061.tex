\documentclass[a4paper,10pt]{article}
\usepackage[utf8]{inputenc}

%opening
%\title{}
%\author{}

\begin{document}

%\maketitle

{\bf Course title: } Convex Optimization.\\

{\bf Course Code: } MATH36061\\

{\bf Course Title: } Convex Optimization\\

{\bf Name of Lecturer: } Martin Lotz\\
 
{\bf Comments: } I am pleased with the overall positive evaluation of this course. This was the second time this course was delivered, and many aspects of the course were new. I was particularly pleased with the feedback on the course material and presentation ("The website is probably one of the best in the school."), and on the delivery of the material.\\

Based on the evaluation, there is room for improvement in the structuring of the problem sessions. In future versions of the course, I will attempt to make them more interactive, to allow for better feedback on the student's work.\\

The few critical comments were concerned mostly with minor formatting issues of the lecture notes (right/left alignment of the lecture notes), with the coordination of problems and lectures, or with overlap in the lectures. The course will be continually developed and optimised, and these comments will all be taken into account. \\

An issue that was raised by one student was the computational content: it was suggested that the course include more programming (maybe half theory, half programming). The problem with this proposal is that, at the moment, I can't assume much programming background as a prerequisite, as very few of the students took the Python course in the second year or have enough equivalent programming experience. The course did include plenty of Python programming examples, access to SageMathCloud, and an introduction to Python (as Jupyter notebook and as video) was included among the course material. However, I feel that including programming into the assessed part of the course (for example, in the form of projects) would be too much of a stretch at the moment. Alternatively, one would have to explicitly set aside problem classes to teach Python, but this would be at the expense of other material, and would require a larger restructuring of the course. Nevertheless, for future versions of the course I will look out for ways to increase the computational aspects.\\

\end{document}
