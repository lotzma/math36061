\documentclass{article}
%=================================================
% Packages
%=================================================

\usepackage{fixltx2e}
\usepackage[usenames,dvipsnames]{xcolor}
\usepackage{fancyhdr}
\usepackage{amsmath,amsfonts,amsbsy,amsgen,amscd,mathrsfs,amssymb}
%\usepackage{amsthm}
\usepackage{subfig}
\usepackage{url}
\usepackage{tikz}
\usepackage{enumitem}
\usepackage{listings}
\lstset{language=Matlab}

\newtheorem{theorem}{Theorem}[section]
\newtheorem{lemma}[theorem]{Lemma}
\newtheorem{corollary}[theorem]{Corollary}
\newtheorem{proposition}[theorem]{Proposition}
\newtheorem{remark}[theorem]{Remark}
\newtheorem{definition}[theorem]{Definition}
\newtheorem{example}[theorem]{Example}
%\newtheorem{remark}[theorem]{Remark}

\numberwithin{equation}{section}

\definecolor{dark-gray}{gray}{0.3}
\usepackage[colorlinks=true]{hyperref}
\hypersetup{urlcolor=Blue}
\hypersetup{citecolor=Black}
\hypersetup{linkcolor=dark-gray}

\usepackage{setspace}
\usepackage{graphicx}
\usepackage{booktabs,longtable,tabu} % Nice tables
\setlength{\tabulinesep}{1pt}
\usepackage{multirow} % More control over tables

% Fonts
%\usepackage{times}
\usepackage{fourier}
\usepackage{bm} % boldmath must be called after the package

%\usepackage{hyperref}
%\usepackage[notcite]{showkeys}
%\usepackage{bbm}  % Loads too many fonts :(
%\usepackage{euscript}
%\usepackage{latexsym}
%\usepackage{color}
%\usepackage[margin=1.25in]{geometry}
%\usepackage{thmtools}
%\usepackage{url}
%\usepackage{citesort}
%\usepackage{supertabular}
%\usepackage[font=small,margin=10pt,labelfont={bf},labelsep={space}]{caption}
%\usepackage{pstricks}

%=================================================
% Paths
%=================================================

\graphicspath{{figures/}}

%=================================================
% Formatting
%=================================================

\sloppy % Helps with margin justification

%%% Further font changes
\newcommand{\lang}{\textit}
\newcommand{\titl}{\textsl}
\newcommand{\term}{\emph}

%%% Equation numbering
\numberwithin{equation}{section} 

%%% Typesetting
\providecommand{\mathbold}[1]{\bm{#1}}  % Must be after 'fourier'
                                % package loads
%%% Annotations
\newcommand{\notate}[1]{\textcolor{red}{\textbf{[#1]}}}

%=================================================
% Theorem environment
%=================================================


%=================================================
% Symbols
%=================================================

%%% Old symbols with new names
\newcommand{\oldphi}{\phi}
\renewcommand{\phi}{\varphi}

\newcommand{\eps}{\varepsilon}
\newcommand{\e}{\varepsilon}

\newcommand{\oldmid}{\mid}
\renewcommand{\mid}{\mathrel{\mathop{:}}} 

%%% New symbols
\newcommand{\defby}{\overset{\mathrm{\scriptscriptstyle{def}}}{=}}
\newcommand{\half}{\tfrac{1}{2}}
\newcommand{\third}{\tfrac{1}{3}}

\newcommand{\sumnl}{\sum\nolimits}

\newcommand{\defeq}{\ensuremath{\mathrel{\mathop{:}}=}} % Definition-equals
\newcommand{\eqdef}{\ensuremath{=\mathrel{\mathop{:}}}} % Equals-definition

%%% Constants
\newcommand{\cnst}[1]{\mathrm{#1}} 
\newcommand{\econst}{\mathrm{e}}
\newcommand{\iunit}{\mathrm{i}}

\newcommand{\onevct}{\mathbf{e}} % All ones vector
\newcommand{\zerovct}{\vct{0}} % Zero vector

\newcommand{\Id}{\mathbf{I}}
\newcommand{\onemtx}{\bm{1}}
\newcommand{\zeromtx}{\bm{0}}

%%% Sets
\newcommand{\coll}[1]{\mathscr{#1}}
\newcommand{\sphere}[1]{\mathsf{S}^{#1}}
\newcommand{\ball}[1]{\mathsf{B}^{#1}}
\newcommand{\Grass}[2]{\mathsf{G}^{#1}_{#2}}
\providecommand{\mathbbm}{\mathbb} % In case we don't load bbm
\newcommand{\Rplus}{\mathbbm{R}_{+}}
\newcommand{\R}{\mathbbm{R}}
\newcommand{\C}{\mathbbm{C}}
\newcommand{\N}{\mathbbm{N}}

% Set operations
\newcommand{\polar}{\circ}
\newcommand{\closure}{\overline}

%%% Real and complex analysis
\newcommand{\abs}[1]{\left\vert {#1} \right\vert}
\newcommand{\abssq}[1]{{\abs{#1}}^2}

\newcommand{\sgn}[1]{\operatorname{sgn}{#1}}
\newcommand{\real}{\operatorname{Re}}
\newcommand{\imag}{\operatorname{Im}}
\newcommand{\pos}{\operatorname{Pos}}
\newcommand{\shrink}{\operatorname{Shrink}}

\newcommand{\diff}[1]{\mathrm{d}{#1}}
\newcommand{\idiff}[1]{\, \diff{#1}}


%\newcommand{\grad}{\nabla} % Conflicts w/SIAM styles?
\newcommand{\subdiff}{\partial}

\newcommand{\argmin}{\operatorname*{arg\; min}}
\newcommand{\argmax}{\operatorname*{arg\; max}}

%%% Probability & measure

\newcommand{\Prob}{\mathbbm{P}}
\newcommand{\Probe}[1]{\Prob\left({#1}\right)}
\newcommand{\Expect}{\operatorname{\mathbb{E}}}

\newcommand{\normal}{\textsc{Normal}}
\newcommand{\uniform}{\textsc{Uniform}}
\newcommand{\erf}{\operatorname{erf}}

\DeclareMathOperator{\rvol}{rvol}
\DeclareMathOperator{\Var}{Var}

%%% Vector and matrix operators
\newcommand{\vct}[1]{\mathbold{#1}}
\newcommand{\mtx}[1]{\mathbold{#1}}
\newcommand{\Eye}{\mathbf{I}}

\newcommand{\transp}{t}
\newcommand{\adj}{*}
\newcommand{\psinv}{\dagger}

\newcommand{\lspan}[1]{\operatorname{span}{#1}}

\newcommand{\range}{\operatorname{range}}
\newcommand{\colspan}{\operatorname{colspan}}

\newcommand{\rank}{\operatorname{rank}}
\newcommand{\nullity}{\operatorname{null}}

%\newcommand{\diag}{\operatorname{diag}}
\newcommand{\trace}{\operatorname{tr}}

%\newcommand{\supp}[1]{\operatorname{supp}(#1)}

\newcommand{\smax}{\sigma_{\max}}
\newcommand{\smin}{\sigma_{\min}}

\newcommand{\nnz}{\operatorname{nnz}}
\renewcommand{\vec}{\operatorname{vec}}

\newcommand{\Proj}{\ensuremath{\mtx{\Pi}}} % Projection operator

%%% Semidefinite orders
\newcommand{\psdle}{\preccurlyeq}
\newcommand{\psdge}{\succcurlyeq}

\newcommand{\psdlt}{\prec}
\newcommand{\psdgt}{\succ}

%%% Mensuration: inner products and norms

% TeX does not like either \newcommand or \renewcommand for these
% two macros.  There is probably a good reason not to use them via
% \def, but I don't know it.  
%\newcommand{\<}{\langle} 
%\newcommand{\>}{\rangle}
\newcommand{\ip}[2]{\left\langle {#1},\ {#2} \right\rangle}
\newcommand{\absip}[2]{\abs{\ip{#1}{#2}}}
\newcommand{\abssqip}[2]{\abssq{\ip{#1}{#2}}}
\newcommand{\tworealip}[2]{2 \, \real{\ip{#1}{#2}}}

\newcommand{\norm}[1]{\left\Vert {#1} \right\Vert}
\newcommand{\normsq}[1]{\norm{#1}^2}

\newcommand{\lone}[1]{\norm{#1}_{\ell_1}}
\newcommand{\smlone}[1]{\|#1\|_{\ell_1}}
\newcommand{\linf}[1]{\norm{#1}_{\ell_\infty}}
\newcommand{\sone}[1]{\norm{#1}_{S_1}}
\newcommand{\snorm}[1]{\sone{#1}}
\DeclareMathOperator{\dist}{dist}

% Fixed-size inner products and norms are useful sometimes
\newcommand{\smip}[2]{\bigl\langle {#1}, \ {#2} \bigr\rangle}
\newcommand{\smabsip}[2]{\bigl\vert \smip{#1}{#2} \bigr\vert}
\newcommand{\smnorm}[2]{{\bigl\Vert {#2} \bigr\Vert}_{#1}}

% Specific norms that are used frequently
\newcommand{\enormdangle}{{\ell_2}}
\newcommand{\enorm}[1]{\norm{#1}}
\newcommand{\enormsm}[1]{\enorm{\smash{#1}}}

\newcommand{\enormsq}[1]{\enorm{#1}^2}

\newcommand{\fnorm}[1]{\norm{#1}_{\mathrm{F}}}
\newcommand{\fnormsq}[1]{\fnorm{#1}^2}

\newcommand{\pnorm}[2]{\norm{#2}_{#1}}
%\newcommand{\snorm}[1]{\norm{#1}_*}

\newcommand{\triplenorm}[1]{\left\vert\!\left\vert\!\left\vert {#1} \right\vert\!\right\vert\!\right\vert} 

% Special cones
\newcommand{\Feas}{\mathcal{F}}
\newcommand{\Desc}{\mathcal{D}}

\newcommand{\sdim}{\delta}
\newcommand{\ddt}[1]{\dot{#1}}
\DeclareMathOperator{\Circ}{Circ}

%%% Differential geometry
\newcommand{\Cinf}{C^{\infty}}
\newcommand{\Tensf}{\mathcal{T}}
\newcommand{\Vbundle}{\mathcal{X}}
\newcommand{\Christ}[3]{\Gamma_{{#1}{#2}}^{#3}}

% Stuff to be sorted in somewhere
\DeclareMathOperator{\Lip}{Lip}


\newcommand{\IR}{\mathbbm{R}}
\newcommand{\veps}{\varepsilon}
\newcommand{\mC}{\mathcal{C}}
\newcommand{\mD}{\mathcal{D}}
\newcommand{\mN}{\mathcal{N}}
\newcommand{\mL}{\mathcal{L}}
\newcommand{\bface}{\overline{\mathcal{F}}}
\newcommand{\face}{\mathcal{F}}
\newcommand{\relint}{\operatorname{relint}}
\newcommand{\cone}{\operatorname{cone}}
\newcommand{\lin}{\operatorname{lin}}
\newcommand{\Gr}{\operatorname{Gr}}
\newcommand{\powerset}{\mathscr{P}}
\newcommand{\hd}{{\operatorname{hd}}}

\newcommand{\conv}{\operatorname{conv}}
\newcommand{\minimize}{\text{minimize}\quad}
\newcommand{\subjto}{\quad\text{subject to}\quad}
\newcommand{\find}[1]{\text{Find }{#1}\quad\longrightarrow\quad}

\newcommand{\comm}[1]{\textcolor{red}{\textbf{[#1]}}}

\usepackage{fancyhdr}
\usepackage{pifont}

% Number of problem sheet
\newcounter{problemSheetNumber}
\setcounter{problemSheetNumber}{1}
\newcommand{\matlabprob}{\ding{100} \ }
\newcommand{\examprob}{\ding{80} \ }
%\setcounter{section}{\theproblemSheetNumber}  
%\renewcommand{\theparagraph}{(\thesection.\arabic{paragraph})}
\newcounter{problems}
\setcounter{problems}{0}
%\setlength{\parindent}{0cm}
\renewcommand{\problem}[1]{\paragraph{(\theproblems)}\addtocounter{problems}{1}\label{#1}}
\ifx\marks\undefined
             \newcommand{\marks}[2][0mm]{\hspace{30mm}\mbox{}\vskip #1\hspace{-30mm}\hfill{\sf [#2 marks]}\\[-\baselineskip]}
\else
             \renewcommand{\marks}[2][0mm]{\hspace{30mm}\mbox{}\vskip #1\hspace{-30mm}\hfill{\sf [#2 marks]}\\[-\baselineskip]}
\fi
\renewcommand{\solution}[1]{\paragraph{Solution (\theproblems)}\addtocounter{problems}{1}\label{#1}}

\pagestyle{fancy}
\lhead{MATH36061}
\chead{Convex Optimization}
\rhead{November 10, 2015}

\begin{document} 
\begin{center}
{\Large {\bf Midterm test}}
\end{center}

\begin{center} 
\emph{Closed book.\quad Attempt all questions.\quad Calculators permitted. \quad 13:00-13:50}\\
{\em Please write your name and student identity number on the front page.}
\end{center}

%-- 
\problem{p0} Determine the order of convergence of each of the following sequences (if they converge at all). You may assume $k\geq 1$.

\begin{equation*}
 \text{(a) } \ x_k = \frac{1}{\sqrt{k}}, \quad \text{(b) } \ x_k = 1+(0.2)^{3^k}, \quad \text{(c) } \ x_k=k^{-k}, \quad \text{(d) } \ x_k = 1
\end{equation*}

\marks[-6mm]{4}

%\newpage
%\mbox{}
%\newpage

\problem{p1} Consider the function on $\R^2$, $f(\vct{x}) = (2x_1+x_2^2)^2$. Show that $\vct{p}=(-1,0)^{\trans}$ is a descent direction at $\vct{x}_0=(0,1)^{\trans}$, and find a step length $\alpha$ that minimizes $f(\vct{x}_0+\alpha p)$. 

\marks[-4mm]{4}

%\newpage
%\mbox{}
%\newpage
%-- 

\problem{p2} Determine, with justification, which of the following functions is convex ($\ln(x)$ refers to the natural logarithm).
\begin{itemize}
 \item[(a)] $\displaystyle f(x)=\ln(x)$ for $x>0$;
 \item[(b)] $\displaystyle f(x)=\frac{1}{x}$ for $x>0$;
 \item[(c)] $\displaystyle f(x,y,z) = z^2-x^2-y^2$ for $x\in \R$;
 \item[(d)] $\displaystyle f(\vct{x}) = \norm{\vct{x}}_1+\norm{\vct{x}}_\infty$.
 \end{itemize}
You may used criteria for convexity seen in the lecture and problem sessions.
\marks[-6mm]{4} 

\problem{p3} Consider the following linear programming problem
\begin{align*}
 \maximize & x_1-x_2 \\
 \subjto & x_1\leq 1\\
	 & x_2\leq 2\\
         & 2x_1+x_2\geq 2
\end{align*}
\begin{itemize}
 \item[(a)] Determine the vertices of the polyhedron of feasible points;
 \item[(b)] Find an optimizer and the optimal value;
 \item[(c)] Write down the dual to this problem.
\end{itemize}

\marks[-6mm]{4} 

\problem{p4} Consider the function
\begin{equation*}
 f(x,y) = \sqrt{1+x^2+y^2}
\end{equation*}
By computing the gradient and Hessian, show that this function is convex and determine the unique minimum. Write down the form of one iteration of Newton's method for this function.
\marks[-4mm]{4}

\newpage

\setcounter{problems}{0}

\solution{so1} 
\begin{itemize}
                \item[(a)] The sequence converges to $0$. We have the identity
\begin{equation*}
 x_{k+1} = \frac{1}{\sqrt{k+1}} = \sqrt{\frac{k}{k+1}} \frac{1}{\sqrt{k}}=\sqrt{\frac{k}{k+1}} x_k,
\end{equation*}
which means that for any fixed constant $c<1$ there is a $k$ such that $1>\sqrt{k/(k+1)}>c$, and therefore $x_{k+1} > cx_k$. It follows that the sequence does not converge linearly (or to any higher order).
\item[(b)] The sequence converges to $1$. We have
\begin{equation*}
 |x_{k+1}-1| = (0.2)^{3^{k+1}} = \left((0.2)^{3^{k}}\right)^3 = |x_k-1|^3,
\end{equation*}
so that the convergence is cubic.
\item[(c)] The sequence converges to $0$. 
We can write $k^{-k}=2^{-k\log(k)}$. For $k\geq 1$,
\begin{equation*}
 x_{k+1} = \frac{1}{(k+1)^{k+1}} = \frac{1}{(k+1)^k \cdot (k+1)} =\frac{1}{2^{k\log(k+1)}(k+1)} \leq \frac{1}{k+1} x_k,
\end{equation*}
which shows that the sequence converges superlinearly.
\item[(d)] The sequence converges to $1$. Moreover, $|x_{k+1}-1|=0\leq M|x_k-1|^p$ for any $M>0$ and $p>0$, so this sequence convergences to any order. 
\end{itemize}

\solution{so2} The gradient is
\begin{equation*}
 \nabla f(x_1,x_2) = \begin{pmatrix}
                      4(2x_1+x_2^2)\\
                      4x_2(2x_1+x_2^2)
                     \end{pmatrix}.
\end{equation*}
At $\vct{x}_0=(0,1)^{\trans}$, $\nabla f(0,1) = (4,4)^{\trans}$. The direction $\vct{p}=(-1,0)^{\trans}$ is a descent direction, if $\ip{\nabla f(\vct{x}_0)}{\vct{p}}<0$. In our case,
$\ip{\nabla f(\vct{x}_0)}{\vct{p}} = -4<0$.
The optimal step length along $\vct{p}$ is the minimizer of
\begin{equation*}
 f(\vct{x}_0+\alpha\vct{p}) = (1-2\alpha)^2.
\end{equation*}
Computing the derivative and setting it to zero, $2\alpha-1=0$, we get the optimal step length $\alpha=1/2$.

\solution{so3} 
\begin{itemize}
 \item[(a)] The function is not convex. The derivative is $1/x$, which for $x>0$ is positive. The second derivative is $-1/x^2<0$.
 \item[(b)] The function is convex. The second derivative is $2/x^3>0$.
 \item[(c)] This function is not convex. The Hessian is given by
 \begin{equation*}
  \begin{pmatrix} -2 & 0 & 0 \\
   0 & -2 & 0\\
   0 & 0 & 2
  \end{pmatrix},
 \end{equation*}
 which is not positive definite.
 \item[(d)] This function is, as the sum of two norms, convex. Precisely, for $\lambda \in [0,1]$,
 \begin{align*}
  \norm{\lambda \vct{x}+(1-\lambda)\vct{y}}_1+\norm{\lambda \vct{x}+(1-\lambda)\vct{y}}_\infty &\leq \lambda\norm{ \vct{x}}_1+(1-\lambda)\norm{\vct{y}}_1+\lambda\norm{ \vct{x}}_\infty+(1-\lambda)\norm{\vct{y}}_\infty\\
  &\leq \lambda(\norm{ \vct{x}}_1+\norm{ \vct{x}}_\infty)+(1-\lambda)(\norm{\vct{y}}_1+\norm{\vct{y}}_\infty).
 \end{align*}

\end{itemize}

\solution{p4} The matrix and the vectors associated to this problem are
\begin{equation*}
 \mtx{A} = \begin{pmatrix}
            1 & 0\\
            0 & 1\\
            -2 & -1
           \end{pmatrix}, \
 \vct{b} = \begin{pmatrix}
            1 \\ 2 \\ -2
           \end{pmatrix}, \
 \vct{c} = \begin{pmatrix}
            1 \\ -1
           \end{pmatrix}.
\end{equation*}

\begin{figure}[h!]
\centering
\begin{tikzpicture}[scale=1.5]\
\draw[color=black, fill=blue!5, thick] (0,2)--(1,2)--(1,0)--(0,2);
\draw[color=black, ->] (0,-0.1)--(0,2.5);
\draw[color=black, ->] (-0.1,0)--(1.5,0);

\draw[color=black] (1,-0.1)--(1,2.1);
\draw[color=black] (-0.1,2)--(1.1,2);
\draw[color=black] (-0.1,2.2)--(1.1,-0.2);

\draw[color=blue,very thick,->] (0,0)--(1,-1);

\draw[color=blue,dashed] (0,-1)--(2,1);

\filldraw[red] (1,0) circle (2pt);

\node (A1) at (0,2)  [label=120:{$\vct{x}_{\{2,3\}}$}] {};
\node (A2) at (1,0)  [label=15:{$\vct{x}_{\{1,3\}}$}] {};
\node (A3) at (1,2)  [label=30:{$\vct{x}_{\{1,2\}}$}] {};
\node (A4) at (1,-1)  [label=0:{$\vct{c}$}] {};
\end{tikzpicture}
\end{figure}

\begin{itemize}
 \item[(a)] To determine the vertices, we can read them off the diagram and then verify analytically that these are indeed the vertices. We get:
 \begin{equation*}
 \vct{x}_{\{1,2\}} = (1,2)^{\trans}, \ \vct{x}_{\{1,3\}} = (1,0)^{\trans}, \ \vct{x}_{\{2,3\}} = (0,2)^{\trans}.
\end{equation*}
 
 Multiplying $\mtx{A}$ with each of these vertices we get
 \begin{equation*}
\mtx{A}\vct{x}_{\{1,2\}} = \begin{pmatrix}
                              1\\2\\-4
                             \end{pmatrix}, \ 
                             \mtx{A}\vct{x}_{\{1,3\}} = \begin{pmatrix}
                              1\\0\\-2
                             \end{pmatrix}, \
  \mtx{A}\vct{x}_{\{2,3\}} = \begin{pmatrix}
                              0\\2\\-2
                             \end{pmatrix}.    
 \end{equation*}
 We see that each of the vertex candidates satisfies all three inequalities, with three as equalities, and we therefore have vertices. 
 \item[(b)] The optimal values can be found among the vertices:
 \begin{equation*}
  \ip{\vct{c}}{\vct{x}_{\{1,2\}}} = -1, \  \ip{\vct{c}}{\vct{x}_{\{1,3\}}} = 1, \ \ip{\vct{c}}{\vct{x}_{\{2,3\}}} = -2. 
 \end{equation*}
As the picture shows, the optimal value is attained at $\vct{x}_{\{1,3\}}$ and the optimal value is $1$.
 \item[(c)] The dual of the problem is
 \begin{equation*}
  \minimize \ip{\vct{b}}{\vct{y}} \quad \subjto \mtx{A}^{\trans}\vct{y}=\vct{c}, \ \vct{y}\geq \zerovct.
 \end{equation*}
Applying this to our data,
\begin{align*}
\minimize & y_1+2y_2-2y_3\\
\subjto & y_1-2y_3 = 1\\
& y_2-y_3 = -1\\
& y_i\geq 0, \quad 1\leq i\leq 3.
\end{align*}
\end{itemize}

\solution{p5} We first compute the gradient and the Hessian of this function.
\begin{equation}\label{eq:grad}\tag{1}
 \nabla f(x_1,x_2) = \frac{1}{\sqrt{1+x^2+y^2}}\begin{pmatrix}
                                  x\\y
                                 \end{pmatrix},
\end{equation}
\begin{equation*}
 \nabla^2 f(x_1,x_2) = \frac{1}{(1+x^2+y^2)^{3/2}} \begin{pmatrix}
                         1+y^2 & -xy\\
                         -xy & 1+x^2
                       \end{pmatrix}.
\end{equation*}
We have a stationary point at $(0,0)$ which is a minimizer, as the function can never fall below $f(0,0)=1$. This means that the Hessian is positive definite at $(0,0)$. There are various ways of verifying that the Hessian is positive definite everywhere, and the function therefore convex. One is direct verification:
\begin{equation*}
 \vct{v}^{\trans} \nabla^2f(x,y)\vct{v}
                                        = v_1^2(1+y^2)-2v_1v_2xy+v_2^2(1+x^2) = v_1^2+v_2^2+(v_1y_1-v_2x_2)^2>0.
\end{equation*}

Newton's method starts with a point $(x_{(0)},y_{(0)})$, and then for every $k\geq 0$, first solves the system of equations
\begin{equation*}
               \frac{1}{(1+x_{(k)}^2+y_{(k)}^2)^{3/2}} \begin{pmatrix}
                         1+y_{(k)}^2 & -x_{(k)}y^{(k)}\\
                         -x_{(k)}y_{(k)} & 1+x_{(k)}^2
                       \end{pmatrix}         
 \begin{pmatrix}
  \Delta x\\ \Delta y
 \end{pmatrix}
= \frac{1}{\sqrt{1+x_{(k)}^2+y_{(k)}^2}}\begin{pmatrix}
                                  x_{(k)}\\y_{(k)}
                                 \end{pmatrix},
\end{equation*}
and then computes
\begin{equation*}
 (x_{(k+1)},y_{(k+1)}) = (x_{(k)},y_{(k)}) + (\Delta x,\Delta y).
\end{equation*}
\end{document} 

%%% Local Variables: 
%%% mode: latex
%%% TeX-master: t
%%% End: 
