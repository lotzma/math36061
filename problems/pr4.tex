\documentclass{article}
% Include macros here
%=================================================
% Packages
%=================================================

\usepackage{fixltx2e}
\usepackage[usenames,dvipsnames]{xcolor}
\usepackage{fancyhdr}
\usepackage{amsmath,amsfonts,amsbsy,amsgen,amscd,mathrsfs,amssymb}
%\usepackage{amsthm}
\usepackage{subfig}
\usepackage{url}
\usepackage{tikz}
\usepackage{enumitem}
\usepackage{listings}
\lstset{language=Matlab}

\newtheorem{theorem}{Theorem}[section]
\newtheorem{lemma}[theorem]{Lemma}
\newtheorem{corollary}[theorem]{Corollary}
\newtheorem{proposition}[theorem]{Proposition}
\newtheorem{remark}[theorem]{Remark}
\newtheorem{definition}[theorem]{Definition}
\newtheorem{example}[theorem]{Example}
%\newtheorem{remark}[theorem]{Remark}

\numberwithin{equation}{section}

\definecolor{dark-gray}{gray}{0.3}
\usepackage[colorlinks=true]{hyperref}
\hypersetup{urlcolor=Blue}
\hypersetup{citecolor=Black}
\hypersetup{linkcolor=dark-gray}

\usepackage{setspace}
\usepackage{graphicx}
\usepackage{booktabs,longtable,tabu} % Nice tables
\setlength{\tabulinesep}{1pt}
\usepackage{multirow} % More control over tables

% Fonts
%\usepackage{times}
\usepackage{fourier}
\usepackage{bm} % boldmath must be called after the package

%\usepackage{hyperref}
%\usepackage[notcite]{showkeys}
%\usepackage{bbm}  % Loads too many fonts :(
%\usepackage{euscript}
%\usepackage{latexsym}
%\usepackage{color}
%\usepackage[margin=1.25in]{geometry}
%\usepackage{thmtools}
%\usepackage{url}
%\usepackage{citesort}
%\usepackage{supertabular}
%\usepackage[font=small,margin=10pt,labelfont={bf},labelsep={space}]{caption}
%\usepackage{pstricks}

%=================================================
% Paths
%=================================================

\graphicspath{{figures/}}

%=================================================
% Formatting
%=================================================

\sloppy % Helps with margin justification

%%% Further font changes
\newcommand{\lang}{\textit}
\newcommand{\titl}{\textsl}
\newcommand{\term}{\emph}

%%% Equation numbering
\numberwithin{equation}{section} 

%%% Typesetting
\providecommand{\mathbold}[1]{\bm{#1}}  % Must be after 'fourier'
                                % package loads
%%% Annotations
\newcommand{\notate}[1]{\textcolor{red}{\textbf{[#1]}}}

%=================================================
% Theorem environment
%=================================================


%=================================================
% Symbols
%=================================================

%%% Old symbols with new names
\newcommand{\oldphi}{\phi}
\renewcommand{\phi}{\varphi}

\newcommand{\eps}{\varepsilon}
\newcommand{\e}{\varepsilon}

\newcommand{\oldmid}{\mid}
\renewcommand{\mid}{\mathrel{\mathop{:}}} 

%%% New symbols
\newcommand{\defby}{\overset{\mathrm{\scriptscriptstyle{def}}}{=}}
\newcommand{\half}{\tfrac{1}{2}}
\newcommand{\third}{\tfrac{1}{3}}

\newcommand{\sumnl}{\sum\nolimits}

\newcommand{\defeq}{\ensuremath{\mathrel{\mathop{:}}=}} % Definition-equals
\newcommand{\eqdef}{\ensuremath{=\mathrel{\mathop{:}}}} % Equals-definition

%%% Constants
\newcommand{\cnst}[1]{\mathrm{#1}} 
\newcommand{\econst}{\mathrm{e}}
\newcommand{\iunit}{\mathrm{i}}

\newcommand{\onevct}{\mathbf{e}} % All ones vector
\newcommand{\zerovct}{\vct{0}} % Zero vector

\newcommand{\Id}{\mathbf{I}}
\newcommand{\onemtx}{\bm{1}}
\newcommand{\zeromtx}{\bm{0}}

%%% Sets
\newcommand{\coll}[1]{\mathscr{#1}}
\newcommand{\sphere}[1]{\mathsf{S}^{#1}}
\newcommand{\ball}[1]{\mathsf{B}^{#1}}
\newcommand{\Grass}[2]{\mathsf{G}^{#1}_{#2}}
\providecommand{\mathbbm}{\mathbb} % In case we don't load bbm
\newcommand{\Rplus}{\mathbbm{R}_{+}}
\newcommand{\R}{\mathbbm{R}}
\newcommand{\C}{\mathbbm{C}}
\newcommand{\N}{\mathbbm{N}}

% Set operations
\newcommand{\polar}{\circ}
\newcommand{\closure}{\overline}

%%% Real and complex analysis
\newcommand{\abs}[1]{\left\vert {#1} \right\vert}
\newcommand{\abssq}[1]{{\abs{#1}}^2}

\newcommand{\sgn}[1]{\operatorname{sgn}{#1}}
\newcommand{\real}{\operatorname{Re}}
\newcommand{\imag}{\operatorname{Im}}
\newcommand{\pos}{\operatorname{Pos}}
\newcommand{\shrink}{\operatorname{Shrink}}

\newcommand{\diff}[1]{\mathrm{d}{#1}}
\newcommand{\idiff}[1]{\, \diff{#1}}


%\newcommand{\grad}{\nabla} % Conflicts w/SIAM styles?
\newcommand{\subdiff}{\partial}

\newcommand{\argmin}{\operatorname*{arg\; min}}
\newcommand{\argmax}{\operatorname*{arg\; max}}

%%% Probability & measure

\newcommand{\Prob}{\mathbbm{P}}
\newcommand{\Probe}[1]{\Prob\left({#1}\right)}
\newcommand{\Expect}{\operatorname{\mathbb{E}}}

\newcommand{\normal}{\textsc{Normal}}
\newcommand{\uniform}{\textsc{Uniform}}
\newcommand{\erf}{\operatorname{erf}}

\DeclareMathOperator{\rvol}{rvol}
\DeclareMathOperator{\Var}{Var}

%%% Vector and matrix operators
\newcommand{\vct}[1]{\mathbold{#1}}
\newcommand{\mtx}[1]{\mathbold{#1}}
\newcommand{\Eye}{\mathbf{I}}

\newcommand{\transp}{t}
\newcommand{\adj}{*}
\newcommand{\psinv}{\dagger}

\newcommand{\lspan}[1]{\operatorname{span}{#1}}

\newcommand{\range}{\operatorname{range}}
\newcommand{\colspan}{\operatorname{colspan}}

\newcommand{\rank}{\operatorname{rank}}
\newcommand{\nullity}{\operatorname{null}}

%\newcommand{\diag}{\operatorname{diag}}
\newcommand{\trace}{\operatorname{tr}}

%\newcommand{\supp}[1]{\operatorname{supp}(#1)}

\newcommand{\smax}{\sigma_{\max}}
\newcommand{\smin}{\sigma_{\min}}

\newcommand{\nnz}{\operatorname{nnz}}
\renewcommand{\vec}{\operatorname{vec}}

\newcommand{\Proj}{\ensuremath{\mtx{\Pi}}} % Projection operator

%%% Semidefinite orders
\newcommand{\psdle}{\preccurlyeq}
\newcommand{\psdge}{\succcurlyeq}

\newcommand{\psdlt}{\prec}
\newcommand{\psdgt}{\succ}

%%% Mensuration: inner products and norms

% TeX does not like either \newcommand or \renewcommand for these
% two macros.  There is probably a good reason not to use them via
% \def, but I don't know it.  
%\newcommand{\<}{\langle} 
%\newcommand{\>}{\rangle}
\newcommand{\ip}[2]{\left\langle {#1},\ {#2} \right\rangle}
\newcommand{\absip}[2]{\abs{\ip{#1}{#2}}}
\newcommand{\abssqip}[2]{\abssq{\ip{#1}{#2}}}
\newcommand{\tworealip}[2]{2 \, \real{\ip{#1}{#2}}}

\newcommand{\norm}[1]{\left\Vert {#1} \right\Vert}
\newcommand{\normsq}[1]{\norm{#1}^2}

\newcommand{\lone}[1]{\norm{#1}_{\ell_1}}
\newcommand{\smlone}[1]{\|#1\|_{\ell_1}}
\newcommand{\linf}[1]{\norm{#1}_{\ell_\infty}}
\newcommand{\sone}[1]{\norm{#1}_{S_1}}
\newcommand{\snorm}[1]{\sone{#1}}
\DeclareMathOperator{\dist}{dist}

% Fixed-size inner products and norms are useful sometimes
\newcommand{\smip}[2]{\bigl\langle {#1}, \ {#2} \bigr\rangle}
\newcommand{\smabsip}[2]{\bigl\vert \smip{#1}{#2} \bigr\vert}
\newcommand{\smnorm}[2]{{\bigl\Vert {#2} \bigr\Vert}_{#1}}

% Specific norms that are used frequently
\newcommand{\enormdangle}{{\ell_2}}
\newcommand{\enorm}[1]{\norm{#1}}
\newcommand{\enormsm}[1]{\enorm{\smash{#1}}}

\newcommand{\enormsq}[1]{\enorm{#1}^2}

\newcommand{\fnorm}[1]{\norm{#1}_{\mathrm{F}}}
\newcommand{\fnormsq}[1]{\fnorm{#1}^2}

\newcommand{\pnorm}[2]{\norm{#2}_{#1}}
%\newcommand{\snorm}[1]{\norm{#1}_*}

\newcommand{\triplenorm}[1]{\left\vert\!\left\vert\!\left\vert {#1} \right\vert\!\right\vert\!\right\vert} 

% Special cones
\newcommand{\Feas}{\mathcal{F}}
\newcommand{\Desc}{\mathcal{D}}

\newcommand{\sdim}{\delta}
\newcommand{\ddt}[1]{\dot{#1}}
\DeclareMathOperator{\Circ}{Circ}

%%% Differential geometry
\newcommand{\Cinf}{C^{\infty}}
\newcommand{\Tensf}{\mathcal{T}}
\newcommand{\Vbundle}{\mathcal{X}}
\newcommand{\Christ}[3]{\Gamma_{{#1}{#2}}^{#3}}

% Stuff to be sorted in somewhere
\DeclareMathOperator{\Lip}{Lip}


\newcommand{\IR}{\mathbbm{R}}
\newcommand{\veps}{\varepsilon}
\newcommand{\mC}{\mathcal{C}}
\newcommand{\mD}{\mathcal{D}}
\newcommand{\mN}{\mathcal{N}}
\newcommand{\mL}{\mathcal{L}}
\newcommand{\bface}{\overline{\mathcal{F}}}
\newcommand{\face}{\mathcal{F}}
\newcommand{\relint}{\operatorname{relint}}
\newcommand{\cone}{\operatorname{cone}}
\newcommand{\lin}{\operatorname{lin}}
\newcommand{\Gr}{\operatorname{Gr}}
\newcommand{\powerset}{\mathscr{P}}
\newcommand{\hd}{{\operatorname{hd}}}

\newcommand{\conv}{\operatorname{conv}}
\newcommand{\minimize}{\text{minimize}\quad}
\newcommand{\subjto}{\quad\text{subject to}\quad}
\newcommand{\find}[1]{\text{Find }{#1}\quad\longrightarrow\quad}

\newcommand{\comm}[1]{\textcolor{red}{\textbf{[#1]}}}

\usepackage{fancyhdr}
%%=================================================
% Packages
%=================================================

\usepackage{fixltx2e}
\usepackage[usenames,dvipsnames]{xcolor}
\usepackage{fancyhdr}
\usepackage{amsmath,amsfonts,amsbsy,amsgen,amscd,mathrsfs,amssymb}
%\usepackage{amsthm}
\usepackage{subfig}
\usepackage{url}
\usepackage{tikz}
\usepackage{enumitem}
\usepackage{listings}
\lstset{language=Matlab}

\newtheorem{theorem}{Theorem}[section]
\newtheorem{lemma}[theorem]{Lemma}
\newtheorem{corollary}[theorem]{Corollary}
\newtheorem{proposition}[theorem]{Proposition}
\newtheorem{remark}[theorem]{Remark}
\newtheorem{definition}[theorem]{Definition}
\newtheorem{example}[theorem]{Example}
%\newtheorem{remark}[theorem]{Remark}

\numberwithin{equation}{section}

\definecolor{dark-gray}{gray}{0.3}
\usepackage[colorlinks=true]{hyperref}
\hypersetup{urlcolor=Blue}
\hypersetup{citecolor=Black}
\hypersetup{linkcolor=dark-gray}

\usepackage{setspace}
\usepackage{graphicx}
\usepackage{booktabs,longtable,tabu} % Nice tables
\setlength{\tabulinesep}{1pt}
\usepackage{multirow} % More control over tables

% Fonts
%\usepackage{times}
\usepackage{fourier}
\usepackage{bm} % boldmath must be called after the package

%\usepackage{hyperref}
%\usepackage[notcite]{showkeys}
%\usepackage{bbm}  % Loads too many fonts :(
%\usepackage{euscript}
%\usepackage{latexsym}
%\usepackage{color}
%\usepackage[margin=1.25in]{geometry}
%\usepackage{thmtools}
%\usepackage{url}
%\usepackage{citesort}
%\usepackage{supertabular}
%\usepackage[font=small,margin=10pt,labelfont={bf},labelsep={space}]{caption}
%\usepackage{pstricks}

%=================================================
% Paths
%=================================================

\graphicspath{{figures/}}

%=================================================
% Formatting
%=================================================

\sloppy % Helps with margin justification

%%% Further font changes
\newcommand{\lang}{\textit}
\newcommand{\titl}{\textsl}
\newcommand{\term}{\emph}

%%% Equation numbering
\numberwithin{equation}{section} 

%%% Typesetting
\providecommand{\mathbold}[1]{\bm{#1}}  % Must be after 'fourier'
                                % package loads
%%% Annotations
\newcommand{\notate}[1]{\textcolor{red}{\textbf{[#1]}}}

%=================================================
% Theorem environment
%=================================================


%=================================================
% Symbols
%=================================================

%%% Old symbols with new names
\newcommand{\oldphi}{\phi}
\renewcommand{\phi}{\varphi}

\newcommand{\eps}{\varepsilon}
\newcommand{\e}{\varepsilon}

\newcommand{\oldmid}{\mid}
\renewcommand{\mid}{\mathrel{\mathop{:}}} 

%%% New symbols
\newcommand{\defby}{\overset{\mathrm{\scriptscriptstyle{def}}}{=}}
\newcommand{\half}{\tfrac{1}{2}}
\newcommand{\third}{\tfrac{1}{3}}

\newcommand{\sumnl}{\sum\nolimits}

\newcommand{\defeq}{\ensuremath{\mathrel{\mathop{:}}=}} % Definition-equals
\newcommand{\eqdef}{\ensuremath{=\mathrel{\mathop{:}}}} % Equals-definition

%%% Constants
\newcommand{\cnst}[1]{\mathrm{#1}} 
\newcommand{\econst}{\mathrm{e}}
\newcommand{\iunit}{\mathrm{i}}

\newcommand{\onevct}{\mathbf{e}} % All ones vector
\newcommand{\zerovct}{\vct{0}} % Zero vector

\newcommand{\Id}{\mathbf{I}}
\newcommand{\onemtx}{\bm{1}}
\newcommand{\zeromtx}{\bm{0}}

%%% Sets
\newcommand{\coll}[1]{\mathscr{#1}}
\newcommand{\sphere}[1]{\mathsf{S}^{#1}}
\newcommand{\ball}[1]{\mathsf{B}^{#1}}
\newcommand{\Grass}[2]{\mathsf{G}^{#1}_{#2}}
\providecommand{\mathbbm}{\mathbb} % In case we don't load bbm
\newcommand{\Rplus}{\mathbbm{R}_{+}}
\newcommand{\R}{\mathbbm{R}}
\newcommand{\C}{\mathbbm{C}}
\newcommand{\N}{\mathbbm{N}}

% Set operations
\newcommand{\polar}{\circ}
\newcommand{\closure}{\overline}

%%% Real and complex analysis
\newcommand{\abs}[1]{\left\vert {#1} \right\vert}
\newcommand{\abssq}[1]{{\abs{#1}}^2}

\newcommand{\sgn}[1]{\operatorname{sgn}{#1}}
\newcommand{\real}{\operatorname{Re}}
\newcommand{\imag}{\operatorname{Im}}
\newcommand{\pos}{\operatorname{Pos}}
\newcommand{\shrink}{\operatorname{Shrink}}

\newcommand{\diff}[1]{\mathrm{d}{#1}}
\newcommand{\idiff}[1]{\, \diff{#1}}


%\newcommand{\grad}{\nabla} % Conflicts w/SIAM styles?
\newcommand{\subdiff}{\partial}

\newcommand{\argmin}{\operatorname*{arg\; min}}
\newcommand{\argmax}{\operatorname*{arg\; max}}

%%% Probability & measure

\newcommand{\Prob}{\mathbbm{P}}
\newcommand{\Probe}[1]{\Prob\left({#1}\right)}
\newcommand{\Expect}{\operatorname{\mathbb{E}}}

\newcommand{\normal}{\textsc{Normal}}
\newcommand{\uniform}{\textsc{Uniform}}
\newcommand{\erf}{\operatorname{erf}}

\DeclareMathOperator{\rvol}{rvol}
\DeclareMathOperator{\Var}{Var}

%%% Vector and matrix operators
\newcommand{\vct}[1]{\mathbold{#1}}
\newcommand{\mtx}[1]{\mathbold{#1}}
\newcommand{\Eye}{\mathbf{I}}

\newcommand{\transp}{t}
\newcommand{\adj}{*}
\newcommand{\psinv}{\dagger}

\newcommand{\lspan}[1]{\operatorname{span}{#1}}

\newcommand{\range}{\operatorname{range}}
\newcommand{\colspan}{\operatorname{colspan}}

\newcommand{\rank}{\operatorname{rank}}
\newcommand{\nullity}{\operatorname{null}}

%\newcommand{\diag}{\operatorname{diag}}
\newcommand{\trace}{\operatorname{tr}}

%\newcommand{\supp}[1]{\operatorname{supp}(#1)}

\newcommand{\smax}{\sigma_{\max}}
\newcommand{\smin}{\sigma_{\min}}

\newcommand{\nnz}{\operatorname{nnz}}
\renewcommand{\vec}{\operatorname{vec}}

\newcommand{\Proj}{\ensuremath{\mtx{\Pi}}} % Projection operator

%%% Semidefinite orders
\newcommand{\psdle}{\preccurlyeq}
\newcommand{\psdge}{\succcurlyeq}

\newcommand{\psdlt}{\prec}
\newcommand{\psdgt}{\succ}

%%% Mensuration: inner products and norms

% TeX does not like either \newcommand or \renewcommand for these
% two macros.  There is probably a good reason not to use them via
% \def, but I don't know it.  
%\newcommand{\<}{\langle} 
%\newcommand{\>}{\rangle}
\newcommand{\ip}[2]{\left\langle {#1},\ {#2} \right\rangle}
\newcommand{\absip}[2]{\abs{\ip{#1}{#2}}}
\newcommand{\abssqip}[2]{\abssq{\ip{#1}{#2}}}
\newcommand{\tworealip}[2]{2 \, \real{\ip{#1}{#2}}}

\newcommand{\norm}[1]{\left\Vert {#1} \right\Vert}
\newcommand{\normsq}[1]{\norm{#1}^2}

\newcommand{\lone}[1]{\norm{#1}_{\ell_1}}
\newcommand{\smlone}[1]{\|#1\|_{\ell_1}}
\newcommand{\linf}[1]{\norm{#1}_{\ell_\infty}}
\newcommand{\sone}[1]{\norm{#1}_{S_1}}
\newcommand{\snorm}[1]{\sone{#1}}
\DeclareMathOperator{\dist}{dist}

% Fixed-size inner products and norms are useful sometimes
\newcommand{\smip}[2]{\bigl\langle {#1}, \ {#2} \bigr\rangle}
\newcommand{\smabsip}[2]{\bigl\vert \smip{#1}{#2} \bigr\vert}
\newcommand{\smnorm}[2]{{\bigl\Vert {#2} \bigr\Vert}_{#1}}

% Specific norms that are used frequently
\newcommand{\enormdangle}{{\ell_2}}
\newcommand{\enorm}[1]{\norm{#1}}
\newcommand{\enormsm}[1]{\enorm{\smash{#1}}}

\newcommand{\enormsq}[1]{\enorm{#1}^2}

\newcommand{\fnorm}[1]{\norm{#1}_{\mathrm{F}}}
\newcommand{\fnormsq}[1]{\fnorm{#1}^2}

\newcommand{\pnorm}[2]{\norm{#2}_{#1}}
%\newcommand{\snorm}[1]{\norm{#1}_*}

\newcommand{\triplenorm}[1]{\left\vert\!\left\vert\!\left\vert {#1} \right\vert\!\right\vert\!\right\vert} 

% Special cones
\newcommand{\Feas}{\mathcal{F}}
\newcommand{\Desc}{\mathcal{D}}

\newcommand{\sdim}{\delta}
\newcommand{\ddt}[1]{\dot{#1}}
\DeclareMathOperator{\Circ}{Circ}

%%% Differential geometry
\newcommand{\Cinf}{C^{\infty}}
\newcommand{\Tensf}{\mathcal{T}}
\newcommand{\Vbundle}{\mathcal{X}}
\newcommand{\Christ}[3]{\Gamma_{{#1}{#2}}^{#3}}

% Stuff to be sorted in somewhere
\DeclareMathOperator{\Lip}{Lip}


\newcommand{\IR}{\mathbbm{R}}
\newcommand{\veps}{\varepsilon}
\newcommand{\mC}{\mathcal{C}}
\newcommand{\mD}{\mathcal{D}}
\newcommand{\mN}{\mathcal{N}}
\newcommand{\mL}{\mathcal{L}}
\newcommand{\bface}{\overline{\mathcal{F}}}
\newcommand{\face}{\mathcal{F}}
\newcommand{\relint}{\operatorname{relint}}
\newcommand{\cone}{\operatorname{cone}}
\newcommand{\lin}{\operatorname{lin}}
\newcommand{\Gr}{\operatorname{Gr}}
\newcommand{\powerset}{\mathscr{P}}
\newcommand{\hd}{{\operatorname{hd}}}

\newcommand{\conv}{\operatorname{conv}}
\newcommand{\minimize}{\text{minimize}\quad}
\newcommand{\subjto}{\quad\text{subject to}\quad}
\newcommand{\find}[1]{\text{Find }{#1}\quad\longrightarrow\quad}

\newcommand{\comm}[1]{\textcolor{red}{\textbf{[#1]}}}

\usepackage{pifont}

% Number of problem sheet
\newcounter{problemSheetNumber}
\setcounter{problemSheetNumber}{4}
\newcommand{\matlabprob}{\ding{100} \ }
\newcommand{\examprob}{\ding{80} \ }
%\setcounter{section}{\theproblemSheetNumber}  
%\renewcommand{\theparagraph}{(\thesection.\arabic{paragraph})}
\newcounter{problems}
\setcounter{problems}{0}
%\setlength{\parindent}{0cm}
\renewcommand{\problem}{\paragraph{(\theproblemSheetNumber.\theproblems)}\addtocounter{problems}{1}}

%\theoremstyle{remark}
%\newtheorem{problem}[problemSheetNumber]{}

\pagestyle{fancy}
\lhead{MATH36061}
\chead{Convex Analysis}
\rhead{October 23, 2016}

\begin{document} 
\begin{center}
{\Large {\bf Problem Sheet \theproblemSheetNumber}}
\end{center}

Problems in Part A will be discussed in class.
Problems in Part B come with solutions and should be tried at home. 

\section*{Part A}

\problem For the following linear programming problems,
\begin{align}\label{eq:ex1}\tag{LP1}
\begin{split}
 \maximize & x_1+2x_2 \\
 \subjto &x_1+x_2\leq 2\\
 & x_1-x_2\leq 1\\
 &x_1\geq -1
\end{split}
\end{align}
\begin{align}\label{eq:ex1}\tag{LP2}
\begin{split}
 \maximize & x_1+x_2 \\
 \subjto
 & x_2-x_1\leq 2\\
 & x_1+x_2\leq 8\\
 &x_1+2x_2\leq 10\\
 &x_1\leq 4\\
 &x_1\geq 0\\
 &x_2\geq 0.
\end{split}
\end{align}

\begin{itemize}
 \item[(a)] Sketch the polyhedron of feasible points and find the vertices;
 \item[(b)] Find a solution, if it exists (you can find the solution visually, but you may use a computer program such as CVXPY in Python or CVX in MATLAB to verify the result).
\end{itemize}

\problem Given a matrix $\mtx{A}\in \R^{m\times n}$ and $\vct{b}\in \R^m$ such that the polyhedron $P=\{\vct{x}\in \R^n\mid \mtx{A}\vct{x}\leq \vct{b}\}$ is not empty and bounded. 
Show that if the optimal value of
\begin{equation*}
 \maximize \ip{\vct{c}}{\vct{x}} \quad \subjto \mtx{A}\vct{x}\leq \vct{b}
\end{equation*}
is finite, it is attained at a vertex $\vct{x}^*$ of $P$. 

\problem Formulate the following optimization problem as a linear programming problem,
\begin{equation*}
 \minimize \norm{\vct{x}}_1 \quad \subjto \mtx{A}\vct{x}=\vct{b},
\end{equation*}
and describe the dual problem.

\problem Show that there exists a vector $\vct{x}\neq \zerovct$ satisfying 
\begin{equation}\label{eq:primal}\tag{P}
 \vct{x}\geq 0, \quad \mtx{A}\vct{x}=\zerovct
\end{equation}
if and only if there is no vector $\vct{y}$ such that
\begin{equation}\label{eq:dual}\tag{D}
 \mtx{A}^{\trans}\vct{y}>\zerovct.
\end{equation}
Give a geometric interpretation of this fact.

\newpage

\section*{Part B}
\problem (Compressive Sensing) Consider a signal $f\colon [0,2\pi]\to \R$
with the property that $f(0)=f(2\pi)$. In practice, one often does not see the whole signal, but only samples certain values at discrete time intervals. It turns out that optimization can help to reconstruct a signal using much fewer samples than commonly thought possible.

\begin{figure}[h!]
 \centering
 \includegraphics[width=0.6\textwidth]{images/subsample_cropped.pdf}
 \caption{Signal sampled at 30 points, and reconstructed from these.}
\label{fig:subsample}
\end{figure}

To understand how the reconstruction from few samples works, we have to look at the Fourier Transform. A periodic function can be written as a Fourier Series
\begin{equation*}
 f(x) = \frac{a_0}{2}+\sum_{n=1}^\infty a_n\cos(nx)+b_n\sin(nx).
\end{equation*}
Setting $c_n=(a_n+ib_n)/2$, $c_{-n}=(a_n-ib_n)/2$ for $n>0$, and $c_0=a_0/2$, the series can also be written in exponential form as as
\begin{equation}\label{eq:fourier-series}\tag{1}
 f(x) = \sum_{n=-\infty}^\infty c_n e^{inx},
\end{equation}
where $e^{ix} = \cos(x)+i\sin(x)$ and $i=\sqrt{-1}$. While this representation involves complex numbers, the resulting function is real due to the way the imaginary parts in the summands combine. 
We can obtain the coefficients $c_n$ in the series~\eqref{eq:fourier-series} by computing
\begin{equation*}
 c_n = \hat{f}(n) := \frac{1}{2\pi} \int_0^{2\pi} f(x) e^{-inx} \ dx.
\end{equation*}
The operation $f\mapsto \hat{f}$ is called the {\em Fourier Transform}. A characteristic feature of many signals is that they are {\em sparse} in the Fourier domain, meaning that only very few summands in the expansion~\eqref{eq:fourier-series} are necessary to describe the signal accurately (in two dimensions, this principle is the basis of the JPEG image compression standard). For the particular function shown in Figure~\ref{fig:subsample}, the representation is 
\begin{equation}\label{eq:signal}\tag{2}
 f(x) = 0.5\sin(5x)+0.5\cos(9x)-\cos(11x)+0.2\sin(13x)+1.7\sin(30x).
\end{equation}

Often we are not so much interested in the analytic expression for the function $f$ but in its values at a discrete set of points
\begin{equation*}
 x_0=0, \quad x_n=2\pi, \quad x_k=\frac{2\pi k}{n}.
\end{equation*}
The goal of reconstructing $f$ then becomes the reconstruction of {\em all} values $f_j=f(x_j)$ from the knowledge of only a few samples $f_k$, $k\in I\subset \{0,\dots,n-1\}$, $|I|=m<n$. 

The function is now represented by a {\em vector} $\vct{f}=(f_0,\dots,f_{n-1})^{\trans}$. The {\em discrete Fourier transform} $\mathrm{DFT}_n(f)$ is a vector $\vct{c}=(c_0,\dots,c_{n-1})^{\trans}$ such that
\begin{equation}\label{eq:discrete-fourier-sum}\tag{3}
 f_j := f(x_j) = \frac{1}{n}\sum_{k=0}^{n-1} c_k e^{ik\frac{2\pi j}{n}}
\end{equation}
for $j=0,\dots,n-1$.
Computing the DFT amounts to solving a linear system
\begin{equation}\label{eq:dft}\tag{4}
 \vct{f} = \mtx{D} \vct{c},
\end{equation}
where the matrix $\mtx{D}$ has the entries $\mtx{D} = (e^{i\frac{2\pi jk}{n}}/n)_{0\leq j,k\leq n-1}$.
The DFT can be computed using the Fast Fourier Transform in $O(n\log n)$ operations. 
Vectors $\vct{f}$ that come from the discretisation of signals like~\eqref{eq:signal} have the property that the {\em Fourier coefficients} $\mathrm{DFT}_n(f)=\vct{c}$ are {\em sparse}, i.e., have only few non-zero entries, see Figure~\ref{fig:fourierex}.

\begin{figure}[h!]
 \centering
 \includegraphics[width=0.6\textwidth]{images/fourier_cropped.pdf}
 \caption{Sparse DFT for signal from Figure~\ref{fig:subsample}.}\label{fig:fourierex}
\end{figure}

The sparsity of DFT vector is the key reason why the following approach for reconstructing $\vct{f}$ from few samples $\vct{f}_I$ works. Consider the optimization problem
\begin{align}\label{eq:l1min}\tag{5}
\begin{split}
 \minimize & \norm{\vct{c}}_1,\\
 \subjto & \mtx{D}_I\vct{c} = \vct{f}_I,
\end{split}
 \end{align}
with the function given as in~\eqref{eq:signal}, where $I\subset[n]$ and $\mtx{D}_I$ is the matrix consisting of the rows indexed by $I$ and $\mtx{f}_I$ the subvector indexed by the entries of $I$ (the red dots in Figure~\ref{fig:subsample}). For example,
\begin{equation*}
 \mtx{D} = \begin{pmatrix} 1 & 2 & 3\\
            2 & 3 & 4\\
            3 & 4 & 5
           \end{pmatrix},
\ I=\{1,3\}, \quad \Longrightarrow \mtx{D}_I = \begin{pmatrix} 1 & 2 & 3\\
            3 & 4 & 5
           \end{pmatrix}
\end{equation*}
In words, we are looking for a vector $\vct{c}$ of minimal $1$-norm that satisfies {\em a small part} of the quadratic system of equations~\eqref{eq:dft}. Once we have such a solution $\hat{\vct{c}}$, the hope is that $\mtx{D}\hat{\vct{c}}=\vct{f}$ gives back the full vector. 

\begin{itemize}
 \item[(a)] Formulate the conditions $\mtx{D}_I\vct{c}=\vct{f}_I$ as {\em linear} constraints involving real numbers. Hint: split $\vct{c}$ and $\mtx{D}_I$ in real and imaginary parts and reformulate the matrix-vector product accordingly.
\item[(b)] Solve the optimization problem~\eqref{eq:l1min} as follows.
\begin{itemize}
\item Set $n=512$ and generate points $x_i=2\pi j/n$ with corresponding values $f_j=f(x_j)$. 
\item Generate the matrix $\mtx{D}$ (in Python, using {\tt numpy.ifft;}) and
choose a random set of $50$ indices to generate a matrix $\mtx{D}_I$ and $\vct{f}_I$. 
\item Using CVX, solve the optimization problem~\eqref{eq:l1min} and compare the computed vector $\hat{\vct{f}}$ with the original $\vct{f}$.
\end{itemize}
\item[(c)] Repeat the experiment in Part (b) with different values of $m$ in order to determine how many samples are necessary to reconstruct the vector $\vct{f}$ accurately.
\item[(d)] Reformulate~\eqref{eq:l1min} as a linear programming problem.
\end{itemize}


\end{document}
